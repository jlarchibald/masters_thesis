


\chapter{Background}

\section{Basics of Ion Inteferometer}

This work includes an overview of ion interferometer experiment and a detailed exposition of the construction of the laser system that drives the stimulated Raman transitions.

The ion interferometer uses a Mach-Zehnder setup to achieve interferometry with $^{87}$Sr+ ions. 

Our 


\section{Historical Precedents}
In this section, we will discuss some related experiments as context for our experiment. This will include historical intellectual forebears and some more modern cousins. 
 \subsection{Interferometry with light}
Matter interferometry borrows many of its foundational concepts and terminology from interferometric light experiments. Though the interference of light has been the subject of serious study for a long time (e.g. Newton's rings), Young's double slit experiment is perhaps the quintessential optics experiment. Young placed two slits and passed light through them. He then observed a pattern of alternating dark and light fringes corresponding to the regions of constructive and destructive interference. 

The diffraction of light in this way is in fact directly analogous to the methods use in some matter-wave interferometry experiments. For example, Alex Kronin uses very small gratings to produce similar patterns \cite{Kronin_RMP}. 

DeBroglie hypothesized that matter exhibits wave-like behavior. Some of the earliest direct confirmation of this was achieved by Davisson and Germer, whose electron scattering experiment was directly analogous in interpretation to Bragg diffraction experiments performed using light. 

\subsection{Coherent Manipulation of Matter}
Matterwave interferometry arose with the development of techniques for the coherent control and measurement of atoms \cite{Kronin_RMP}. Rabi was among the first to use rf resonances to coherently control the internal states of molecules\footnote{I think this is right}. Ramsey's work in creating long-lived superpositions and the method of separate oscillatory fields was similarly seminal \cite{Kronin_RMP}. 
More recently, atom interferometers have been used for many different types of experiments. These have provided high-precision measurements of the fine structure constant \footnote{I'll have to find some citations. This is basically just a list of things I saw at conferences that I thought were cool}, investigations of gravitational effects and placed ever tighter limits on potential deviations from Standard Model physics. 
%\subsection{Charged Particle Atomic Physics}

\subsection{Laboratory-based electromagnetic tests}
One possible application of our apparatus is that it can be used to detect small electric fields in a setup like the one famously used by Cavendish in his early electrostatic experiments\footnote{I am not refering to the famous torsion balance gravitational experiment.} Cavendish effectively measured fundamental electrostatic constants by measuring the electric filed inside a charged, conducting enclosure. By demonstrating that the electric fields were 0 regarldless of the charge on the enclosure, Cavendish was able to deduce that the repulsion between like electric charges falls off in proportion to the inverse of the square of the distance between them. \footnote{todo: cite this guy later maybe:  http://geometrydude.wordpress.com/2012/12/22/cavendishs-phenomenological-derivation-of-coulombs-law/}. 


Our experiment is similar to Cavendish's experiment in two ways: First, our experiment will have a very similar setup as Cavendish's. The ion interferometer provides a sensitive way to measure electric fields inside a conducting shell that will be charged and discharged.  

The second way in which our experiment is related to Cavendish's is that it seeks to shed light on fundamental electromagnetic phenomena. Our our experiment seeks to measure fundamental electrical constants. Cavendish was able to measure the $1/r^2$ dependence in what eventually became known as Coulomb's law. Our experiment hopes to place a limit on the photon rest mass.\footnote{Our experiment will also produce results that may be reported as $\epsilon$ in the $1/r^{2+\epsilon}$ generalization of Coulomb's law.}
%footnote{who am I to say who did what first?}\cite{jackson} \footnote{it'd be nice to get Cavendish's original paper} 
%I don't really know if the finite conductivity of the shell matters for our experiment. 
%can we characterize copper in some way? I mean by taking time-dependent measurements of the dispersion of charge within the cavity? 
In fact, Cavendish's experiment directly inspired our experiment\footnote{This is a piece of lab oral history that I wanted to preserve}. Brian Neyenhuis, whose time working in Dallin's lab overlapped with mine, came across a practice GRE question about the best laboratory measurements to date confirming Coulomb's law. It turned out that these were a variation on Cavendish's method. They key limiting factor in all cases was the experimenter's ability to measure small electric fields inside the conducting shell. Brian started investigating the use of atomic physics techniques to improve these measurements. Soon he and Dallin had fleshed out their idea for creating an ion interferometer.  

According to Ref.\ \cite{jackson}. Many of the more recent tests of electrostatics were performed using setups similar to Cavendish's. 

%todo: find that paper that cites our experiment 
There are many other experiments along these lines as detailed in \cite{PhotonMassSurvey}. 

\subsection{More recent antecedent}
Electron interferometry using gratings has been achieved. \footnote{cite!!} Furthermore, ion interferometry has been achieved only using physical gratings. One of our experiment's main advantages is that its beamsplitters use no physical gratings

There are atomic clocks that use long-lived coherent superpositions of internal states of ions.\footnote{todo:cite}. However, The ions themselves do not enclose spacelike areas over the course of their path.

Coherent manipulation has also (thankfully!) been achieved in many labs \footnote{Todo:CITE!}. For example, neutral Sr has been trapped and cooled by XXXXX. 
Other experiments involving Sr TODO: add the other experiments.

The interest in $^{87}$Sr+ has been helpful not least because there are good numerical estimates of various transition parameters \cite{safronovaTheory}


