\section{Historical Precedents}
In this section, we will discuss some related experiments as context for our experiment. This will include historical intellectual forebears and some more modern cousins. 
 \subsection{Interference effects involving light}
Wave phenomena are myriad. Matter interferometry borrows many of its foundational concepts and terminology from interferometric light experiments.
The wave nature of light has been the subject of serious study for a long time.
Grimaldi observed diffraction early on. Hooke and Huygens and Newton. \footnote{http://www.thestargarden.co.uk/NewtonAndLight.html}\footnote{shoot--this stuff is turning out to be hard to pin down. It seems that despite everything (Newton's rings and all), Newton wasn't really hip to waves and seems to have accomplished a lot modeling things as corpuscular little particles with inherent colors that reflect and refract differently.}

Young's double slit experiment played an important role in the acceptance of a wave model of light. Young placed two slits and passed light through them. He then observed a pattern of alternating dark and light fringes corresponding to the regions of constructive and destructive interference. This represents a species of interferometry since there are two possible paths and their result depends on the relative phase shift between them. 

However, the word, "interferometer" as typically used refers to an experiment where light is split and then recombined in some way. 

Mach and Zehnder published work on their interferometers in 1892 and 1891 respectively \cite{zehnder}\cite{mach}. Their names are now synomous with this basic layout (see, for example, Fig.~\ref{mach-zehnder-fig}.  

The Michelson-Morley interferometer was an important optical interferometer. Their experiment involved sending light along two arms and retroreflecting the light back so that it was combined by the same beam splitter that separated it. Their experiment was intended to look for aether drift, and their famous null result was later interpreted as important early evidence of special relativity. 

\subsection{Experiments involing diffraction of matterwaves}
DeBroglie hypothesized that matter exhibits wave-like behavior. Some of the earliest direct confirmation of this was achieved by Davisson and Germer\cite{davisson_and_germer}, whose electron scattering experiment was directly analogous in interpretation to Bragg diffraction experiments performed using light. The first atomic diffraction experiment was by Estermann and Stern \cite{esterman_stern}\cite{Kronin_RMP}. Diffraction of matterwaves has also been achieved with larger pieces of matter, including buckyballs\cite{C60_interferometry} and other large molecules \cite{large_molecule_interferometry}.

\subsection{True separated-beam matterwave interferometers}
Another class of experiments involves the use of physical diffraction gratings to separate, redirect and recombine matter waves from various types of massive particles. This was first achieved by \cite{electronGrating} for electrons and \cite{neutronGrating} for neutrons. The first atomic interferometers of this type were by Ref.~\cite{pritchard1991} and \cite{youngDoubleSlit_Carnal_Mlynek}. These experiments use grating and slits similar to some of the early optics experiments. However, all four of these experiments used gratings both to split, redirect and recombine the travelling matterwaves. This is presumably because this is a higher resolution technique. Essentially, these were so-called "white-light" interferometers where the resolution of any hypothetical fringes would be washed out by the large distribution of particle wavelengths \footnote{Actually, Dallin better take a close look here.} 


\subsection{Coherent Manipulation of Matter using the internal states of atoms}
Matterwave interferometry arose with the development of techniques for the coherent control and measurement of atoms \cite{Kronin_RMP}. Rabi was among the first to use rf resonances to coherently control the internal states of molecules\cite{RabiOriginal}\cite{Kronin_RMP}. 

Ramsey's work in creating long-lived superpositions and the method of separate oscillatory fields was similarly seminal \cite{Kronin_RMP}\cite{Ramsey_original}. 
Bord\'e built \cite{borde_interferometer} the first of these. 
More recently, atom interferometers have been used for many different types of experiments. These have provided high-precision measurements of the fine structure constant, investigations of gravitational effects and placed ever tighter limits on potential deviations from Standard Model physics \cite{Kronin_RMP}.
%\subsection{Charged Particle Atomic Physics}

\subsection{Laboratory-based electromagnetic tests}

One possible application of our apparatus is that it can be used  to detect small electric fields in a setup like the one famously used by Cavendish in his early electrostatic experiments. Cavendish effectively measured fundamental electrostatic constants by measuring the electric filed inside a charged, conducting enclosure. By demonstrating that the electric fields were 0 regardless of the charge on the enclosure, Cavendish was able to deduce that the repulsion between like electric charges falls off in proportion to the inverse of the square of the distance between them. \footnote{todo: cite this guy later maybe:  http://geometrydude.wordpress.com/2012/12/22/cavendishs-phenomenological-derivation-of-coulombs-law/}. 

The ion interferometer is Our experiment is similar to Cavendish's experiment in two ways: First, one possible application of the ion interferometer is that it could be placed inside a conductive shell whose electric potential is varied. have a very similar setup as Cavendish's. The ion interferometer provides a sensitive way to measure electric fields inside a conducting shell that will be charged and discharged. 

The second way in which our experiment is related to Cavendish's is that it seeks to shed light on fundamental electromagnetic phenomena. Our our experiment seeks to measure fundamental electrical constants. Cavendish was able to measure the $1/r^2$ dependence in what eventually became known as Coulomb's law. Our experiment hopes to place a limit on the photon rest mass.\footnote{Our experiment will also produce results that may be reported as $\epsilon$ in the $1/r^{2+\epsilon}$ generalization of Coulomb's law.}
%footnote{who am I to say who did what first?}\cite{jackson} \footnote{it'd be nice to get Cavendish's original paper} 
%I don't really know if the finite conductivity of the shell matters for our experiment. 
%can we characterize copper in some way? I mean by taking time-dependent measurements of the dispersion of charge within the cavity? 
In fact, Cavendish's experiment directly inspired our experiment\footnote{This is a piece of lab oral history that I wanted to preserve}. A former student in the lab, Brian Neyenhuis, came across a practice GRE question about the best laboratory measurements to date confirming Coulomb's law. It turned out that these were a variation on Cavendish's method. They key limiting factor in all cases was the experimenter's ability to measure small electric fields inside the conducting shell. Brian started investigating the use of atomic physics techniques to improve these measurements. Soon he and Dallin had fleshed out their idea for creating an ion interferometer \cite{NeyenhuisIon}\cite{christensen_arxiv_calcs}.  

According to Ref.\ \cite{jackson}. Many of the more recent tests of electrostatics were performed using setups similar to Cavendish's. 

%todo: find that paper that cites our experiment 
There are many other experiments along these lines as detailed in \cite{PhotonMassSurvey}. 

\subsection{The $^{87}$Sr+ interferometer's place}
the 


There are atomic clocks that use long-lived coherent superpositions of internal states of ions.\footnote{todo:cite}. However, The ions themselves do not enclose spacelike areas over the course of their path.

Coherent manipulation has also (thankfully!) been achieved in many labs \footnote{Todo:CITE!}. For example, neutral Sr has been trapped and cooled by XXXXX. 
Other experiments involving Sr TODO: add the other experiments.

The interest in $^{87}$Sr+ has been helpful not least because there are good numerical estimates of various transition parameters \cite{safronovaTheory}

The diffraction of light in this way is in fact directly analogous to the methods use in some matter-wave interferometry experiments. For example, Alex Cronin uses very small gratings to produce similar patterns \cite{Kronin_RMP}. 

