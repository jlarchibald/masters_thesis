\section{Precedents}
In this section, I will discuss some related experiments. These experiments represent the intellectual heritage of our work.
%have influenced and built the foundation for our work. 
A few of the ideas that are central to our project are over 100 years old. Other experiments to which we are indebted are of much more recent vintage. These more recent experiments also pioneered many of the technologies, techniques, and tools that we use. 
%Our experiment also owes a great intellectual debt to more recent experiments. 
%These provided the specific technologies, techniques, and tools that we use. 
 \subsection{Interference effects involving light}
Matterwave interferometry borrows many of its foundational concepts and terminology from interferometric light experiments.
The wave nature of light has been the subject of serious study for a long time. It was first discussed as a way to explain diffraction.
Grimaldi observed diffraction early on. Hooke, Huygens and Newton also did well-known early experimental and theoretical work involving the propagation of light and its interference phonomena. 
%\footnote{http://www.thestargarden.co.uk/NewtonAndLight.html}
%\footnote{shoot--this stuff is turning out to be hard to pin down. It seems that despite everything (Newton's rings and all), Newton wasn't really hip to waves and seems to have accomplished a lot modeling things as corpuscular little particles with inherent colors that reflect and refract differently.} 
However, the word, ``interferometer'' usually refers to an experiment where light rays are split and then recombined in some way after travelling along two distinct paths. 
Young's double slit experiment, which played an important role in the acceptance of a wave model of light, could be viewed as an interferometer. 
%His experiment also played an important role in the acceptance of a wave model of light.
Young placed two slits and passed light through them. He then observed a pattern of alternating dark and light fringes corresponding to the regions of constructive and destructive interference. This represents a type of interferometry since, for any given point on the output plane, there are two distinct, possible paths that the light could have taken to get there (one path passes through each slit). The output depends on the relative phase shift between the light travelling along each of the two paths. 

Mach and Zehnder published work on their interferometers in 1892 and 1891 respectively \cite{mach}\cite{zehnder}. Their names are now synomous with the basic interferometer layout that they pioneered (see, for example, Fig.~\ref{mach-zehnder-fig}).  ``Mach-Zehnder'' interferometer now refers specifically to any interferometer that uses different beam splitters to split and combine the wave and that encloses area. By contrast, a Michelson-Morley type interferometer uses the same beam splitter to both split and recombine the light. Thus, such an interferometer encloses no area. 
%Theirs was also an important optical interferometer. Their experiment involved sending light along two arms and retroreflecting the light back so that it was combined by the same beam splitter that had separated it. 
The original Michaelson-Morley interferometer was also among the most influential light interferometers. The goal of Michaelson's and Morley's original experiment was to look for ether drift and their famous null result was later interpreted as important early evidence of special relativity.

\subsection{Experiments involing diffraction of matter waves}
DeBroglie hypothesized that matter exhibits wave-like behavior. Some of the earliest direct confirmation of this was achieved by Davisson and Germer\cite{davisson_and_germer}, whose electron scattering experiment was somewhat analogous to Bragg diffraction experiments performed using light. The first atomic diffraction experiment was by Estermann and Stern \cite{esterman_stern}\cite{Kronin_RMP}. Diffraction of matterwaves has also been the basis for a number of interferometers that use a series of physical diffraction gratings to separate, redirect and recombine matter waves of various types of massive particles. This was first achieved by Marton\cite{electronGrating} for electrons and Rauch\cite{neutronGrating} for neutrons. The first atomic interferometers of this type were created by Pritchard\cite{pritchard1991} and Mlynek\cite{youngDoubleSlit_Carnal_Mlynek}. These experiments use gratings and slits similar to some of the early optics experiments. Experiments of this type have also been done with larger pieces of matter, including buckyballs\cite{C60_interferometry} and other large molecules \cite{large_molecule_interferometry}.
%Cite \cite{BatelaanAB} ? Not sure. Doesn't fit that well. 

The development of techniques for the coherent control and measurement of atoms was a major boon to matterwave interferometry\cite{Kronin_RMP}. Rabi was among the first to use rf resonances to coherently control the internal states of molecules\cite{RabiOriginal}\cite{Kronin_RMP}. 

Ramsey's work in creating long-lived superpositions and the method of separated oscillatory fields was similarly seminal \cite{Kronin_RMP}\cite{Ramsey_original}. 
Bord\'e built \cite{borde_interferometer} the first separated beam atom interferometer using light to split the wavefunction. 
More recently, atom interferometers have been used for many different types of experiments. These have provided high-precision measurements of the fine structure constant\cite{WichtFineStructure}\cite{WeissFineStructure}\cite{GibbleFineStructure}, investigations of gravitational effects \cite{mullerIsotropyGR}\cite{KasevichGravWaves} and ever tighter limits on potential deviations from Standard Model physics\cite{mullerLorentzInvarianceElectrodynamics}  \cite{Kronin_RMP}\cite{KasevichInertial}.
%\subsection{Charged Particle Atomic Physics}

Kasevich and Chu pioneered the use of Raman transitions as atomic beam splitters\cite{kasevichChu1991}. The matterwave interferometers that they built are the most similar experiments to the ion interferometer that we are constructing. 

The project also relies on trapping and cooling of neutral $^{87}$Sr, which was previously achieved and documented in Ref.~\cite{kurosu_trap_sr}. Sr has been used in other experiments, including atomic clocks \cite{ludlow_science}.


\subsection{Laboratory-based electromagnetic tests}

One possible application of our apparatus is to use the $^{87}$Sr$^+$ ions to detect small electric fields in a setup like the one famously used by Cavendish in his early electrostatic experiments. Cavendish effectively measured fundamental electrostatic constants by measuring the electric field inside a charged, conducting enclosure. By demonstrating that the electric fields inside the enclosure were zero regardless of the charge on the enclosure, Cavendish was able to deduce that the repulsion between like electric charges falls off in proportion to the inverse of the square of the distance between them \cite{geodude}. Many of the more recent tests of electrostatics were performed using setups similar to Cavendish's\cite{jackson}.  In fact, the main improvement that these experiments made over Cavendish's was that they used modern electronics rather than suspended pith balls to measure the electric field inside the conductor.

The ion interferometer relates to Cavendish's experiment in three ways: First, the ion interferometer could be used as the electric-field-detecting component in a Cavendish-type experiment. The ion interferometer will be very sensitive to electric fields and it could be placed inside a conductive shell whose electric potential is varied. In fact, it would represent a significant improvement over current methods for measuring small electric fields in Cavendish-like experiments \cite{NSFprop}. 

The second way in which our project is related to Cavendish's experiment is that it seeks to shed light on fundamental electromagnetic phenomena. If enclosed in a conducting shell (as mentioned in the previous paragraph), our experiment could be used to measure fundamental electrical constants. Cavendish was able to measure the $1/r^2$ dependence in what eventually became known as Coulomb's law. We hope that our experiment will be able to place limits on the photon rest mass.\footnote{Our experiment will also produce results that may be reported as $\epsilon$ in the $1/r^{2+\epsilon}$ generalization of Coulomb's law.}
%footnote{who am I to say who did what first?}\cite{jackson} \footnote{it'd be nice to get Cavendish's original paper} 
%I don't really know if the finite conductivity of the shell matters for our experiment. 
%can we characterize copper in some way? I mean by taking time-dependent measurements of the dispersion of charge within the cavity? 

Third, Cavendish's experiment directly inspired our experiment.
%\footnote{This is a piece of lab oral history that I wanted to preserve}. 
A former student in the lab, Brian Neyenhuis, came across a practice GRE question%\footnote{A real GRE question, possibly?}
 about the best laboratory measurements to date confirming Coulomb's law, most of which are a variation on Cavendish's method. Neyenhuis started investigating the use of atomic physics techniques to improve these measurements when he learned that the key limiting factor in all cases was the experimenter's ability to measure small electric fields inside the conducting shell. Soon Neyenhuis and Durfee, along with another student, Christensen, had fleshed out their idea for creating an ion interferometer \cite{christensen_arxiv_calcs} \cite{NeyenhuisIon}.  

Our project fits within a broader effort to perform laboratory tests of fundamental electromagnetic quantities. A good survey of the other experiments with potential applications for fundamental electromagnetic measurements can be found in Ref.\,\cite{PhotonMassSurvey}. 

%There are many experiments using long-lived coherent superpositions of internal states of ions, including atomic clocks\cite{Rosenband_ion_clock}.
%\footnote{Actually, I don't know why they use the ions in the clocks, but my guess is that they } 
%and possibly other experiments.
%\footnote{weak, but I don't know what else to put here. Maybe this isn't such a valuable paragraph.}.
%There are myriad other experiments involving Sr, %\footnote{Probably there is at least one other?}.
%The interest in $^{87}$Sr$^+$ has been helpful not least because there are good numerical estimates of various transition parameters \cite{safronovaTheory}
