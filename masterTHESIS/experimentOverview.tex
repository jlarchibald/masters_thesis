\chapter{Ion Interferometer Guided Tour}
An ion to be used in the interferometer goes through three distinct stages: the trapping and cooling stage, ionization and, finally, the interferometry stage. 

%We must insert a figure right here if possible. 

\section{Trapping of Neutral Strontium and Low Velocity Intense Source (LVIS)}

First, we must obtain a slow-moving beam of ionized $^{87}$Sr$+$. This is accomplished in two steps. 

We trap the $^{87}$Sr in a magneto-optical trap (MOT) using the standard techniques as outlined in Ref.\ \cite{cjeDiss}. 

The dominant cooling effect in the MOT is Doppler cooling. The trap consists of 6 red-detuned laser beams originating from the 460.8 nm laser system described in \cite{cjeDiss}. These point from each of six directions towards the center of the MOT. 

The 5s$^2$ $^1$S$_0$ to 5s5p $^1$P$_1$ transition is used to cool the atoms. This process selects the 87u isotope of Strontium \footnote{Further rejection of other Sr isotopes occurs because the Raman beams will not drive their transitions.}
The trapping is achieved using a pair of permanent rare-earth magnets. The Zeeman shift of the levels addressed by the the lasers traps them within a small range. The amount of trapped atoms is enhanced by our use of a Zeeman slower. 

We create an LVIS \cite{cjeDiss} following techniques developed by \cite{LVIS}. One of the mirrors in in the trap has a small hole drilled in it.\footnote{This hole was drilled in-house and the details of how we avoided damage and tests that verify the integrity of its optical surface can be seen in \cite{cjeDiss}} This allows some of the atoms to escape the trap. These atoms have a relatively narrow distribution of velocities. \footnote{Probably cite some other reference here--whoever invented/talked about the LVIS}

The atoms that drift out of the trap are then accelerated by a sort of Zeeman acceleration effect \footnote{i.e. the same as the effect exploited in a Zeeman slower, but they're not slowing down. The magnetic field gradients that cause this are created by the same magnets that are used to make the gradient in the middle of the trap.}

\section{Ionization of Strontium}
 
The next step is to selectively ionize the $^{87}$Sr to produce $^{87}$Sr$+$. This is accomplished using a pair of resonant transitions, which place the electron in a state in an autoionizing state whose energy is higher than the lowest energy states in the continuum. 

First, we stimulate the 5s$^2$ $^1$S$_0 \rightarrow$ 5s5p $^1$P$_1$ transition, which, conveniently, is the same 461 nm transition used to cool the atom. The second transition is the 5s5p $^1$P$_1\rightarrow$5p$^2$ $^1$D$_2$ transition. The $^1$D$_2$ state is 57 meV above the ionization threshold \cite{NSFprop}. This can be accomplished with a 408 nm laser. This can be achieved using GaN diode lasers \footnote{Maybe InGaN? It is the same diode laser technology used in Blu-ray players.}

%\section{Acceleration of Beam}

The ions, which are at this point already travelling at some speed are then accelerated in an electric field produced by two pieces of copper held at constant potential. This gives us freedom to control the speed of the atoms. 


\section{Interferometry}

The interferometry is then achieved. The interferometer is set up in a Mach-Zehnder configuration. \footnote{actually, are all interferometers like this? I'm not sure what makes ours a Mach Zehnder. Ramsey Borde, which it's not, is a subset of Mach Zehnder? }

The crucial feature of any interferometer is the splitting and recombining of waves in such a way that the relative phase difference between the two paths may be indirectly observed. In our interferometer, we achieve interference of the atomic wave function of our $^{87}$Sr$^+$ ions. 
%the waves that are interfered are the quantum mechanical matter waves of our atoms. I borrowed the words "atomic wave function" phrase from chu and kasevich

In order to produce two distinct paths, we uses stimulated Raman transitions as done by Ref.\ \cite{kasevichChu1991}. We use stimulated Raman transitions. This is a two photon process whereby we drive the atoms from one state to another using a third state as an intermediate state. We have 3 pairs of overlapping 408 nm lasers in our apparatus and each of these pairs is tuned to induce a stimulated Raman transition between the two $^2$S$_{1/2}$ ($5s$) ground states corresponding to $F=4$ and $F=5$. The transition uses the $^2$P$_{3/2}$ ($5p$) state as an intermediate state. 

%The stimulated Raman transition process is one whereby we 

The essential analysis will be saved for Chapter \ref{ChapterAboutTheAtoms}.


%trust yourself and know that you know it. and know that it doesn't have to be much better than it will be to be acceptable. 

%Is it true that at the end, we can distinguish between the F=4 and F=5 ground state of the Strontium? 

\section{Scope of this work}
My work includes only the construction of the 408 nm laser system that stimulates the Raman transitions in the interferometer. 

