 \chapter{Injection Locking}\label{InjectionLockingChapter}
 %(mode matching, measuring photocurrent from lasers, details of optimizing the isolators, some important caveats about not letting the wrong light couple back into the other laser. 

Injection locking two slave lasers was necessary to get more power than would have been available from just the shifted beams from the master laser alone. 
In order to achieve injection locking, we must couple the first order diffracted light that came out of the AOM into the laser cavities of each of the slave lasers. This is the first step in seeding the slave lasers with light from the master laser. The second step is to adjust the current and temperature of the slave lasers so that they achieve prolonged, single-mode operation in a mode that is in tune with the seed light. 

\section{Theory of operation}

%why did this work at all? How did we know how much light would work? 

The laser operates on the principle of stimulated emission of radiation. In a non-injection-locked laser, the laser's output is determined by the interplay of several factors, including the resonant modes of the laser cavity and the spectral characteristics of the laser gain medium.
``Mode competition'' is the process whereby the power circulating in the laser concentrates in certain, more dominant modes at the expense of others. Since the gain saturation of the diode laser is determined largely by the rate at which electron-hole pairs can be created, it is possible for one mode to dominate the other modes by reaching not only self-saturation, but cross-saturating the gain medium for the other modes\cite{RPPhotonicsEncyclopediaAndBuyersGuide}. This results in essentially single mode operation. 

Therefore, intuitively, the goal of injection locking is to merely couple enough light into a suitable mode of the slave laser so that this mode has a slight advantage over the other modes. 

%did this stuff even work? 
\section{Initial slave configuration}
\label{initialSlaveConfiguration}


I selected the diodes for the slave lasers in the same manner in which I selected the master laser diode. I used the wavelength-selected diodes with the data I took on the Ocean Optics spectrometer. The setup and configuration of the slaves was very similar to what I did with the master laser, including all the part numbers and control electronics as described in Chapter\,\ref{masterLaserDiodes}. The slave lasers, however, have no diffraction grating. Rather than using grating feedback, the slave lasers achieve improved stability and wavelength control because they are injected with shifted light from the AOM, whose stability is determined by the stability of the master laser and the rf oscillator that drives the AOM. 

%can I use the diffraction grating with seed light? 


%Notice that there was something about stray light reflections that we had to be careful of, I think. 

\section{Matching the mode properly}
I optimized coupling to the laser diode by using the following technique: First, I turned off the slave lasers and disconnected them from their current supplies. Next, I used the slave laser diode as a photodiode. I coupled light from the AOM into the laser and measured the photocurrent produced by the laser. Better coupling between the beam from the AOM and the laser results in higher photocurrent coming out of the laser. In this way, I can conveniently align the optics. 
%In order to couple to the slave lasers, we used the following technique: was that we detached the lasers from their current supplies and then used the lasers 
%For Slave 1, the half-intensity angle was rated to be 9$^\circ$ in the direction parallel to the polarization. and 19$^\circ$ in the perpendicular direction. 

I also adjusted the temperatures of the lasers in order to maximize the range of currents over which they would stay injected. 
When slave 2 was injection locked, the current produced when coupling the light in from the AOM was 8.2$\mu$A. The current at which it was working was 71.752 mA %(27292 counts on the digital controller) 
and the temperature was found to be 311.309K. %(59112 counts on the digital controller)\footnote{Lab Notebook section V page 66}.
Slave 1 worked properly at XXXXXXX TODO: FIND THIS.

The amount of power being sent towards each slave (though, not necessarily being coupled in) is $\sim$90$\mu W$. The slaves typically produce 48 mW to 63 mW.
