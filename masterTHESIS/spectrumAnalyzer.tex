\section{Spectrum Analyzer}

The spectrum analyzer consists of a 10 cm cavity with two mirrors XXXXXXX FIND OUT WHICH ONES. 

One of the mirrors is mounted on a Thorlabs kinematic piezo mount and attached to a piezo driver designed specifically for spectrum analyzers.

The cavity is designed to be nearly semiconfocal\footnote{I used to think that this was because the transverse modes of a semiconfocal cavity overlap. However, it seems that a concentric cavity would be more likely to have this property. Oh wait. I get it. The semiconfocal cavity has all its transverse modes very small.} This is because the transverse modes of a semiconfocal cavity correspond to longitudinal modes with comparable lengths. This ensures that it is relatively easy to couple the laser to the cavity since we are relatively insensitive to which of the transverse modes we are coupling to. 

%\footnote{How do I know whether I'm semiconfocal for all practical purposes? I mean, somehow lower finesse should make me less sensitive to small variations in the cavity length. Furthermore, are there models where the errors never really go away? You know? I guess I mean where it doesn't just become part of the noise. I suppose chaos is one example of this--where arbitrarily high precision is needed to predict even the macroscopic behavior of the system. Is any aspect of this system chaotic?}

The alignment procedure was as follows:

The entry mirror was aligned so as to be retroreflecting the incoming beam. Later, The rear mirror was adjusted so that beam reflected back. 
