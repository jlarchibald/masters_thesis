\chapter{The Atoms} \label{ChapterAboutTheAtoms}
\section{Overview of relevant atomic transitions}

The system is designed to stimulate Raman transitions between the hyperfine $^2$S$_{1/2}$ ($5s$) ground states corresponding to $F=4$ and $F=5$. The intermediate state is the $^2$P$_{3/2}$ ($5p$) state. 

In Ref.\ \cite{cjeDiss}, this calculation is performed, but some of the details are left mysterious. We hope to reproduce this calculation with better details.  

\section{Summary of First Approach to This Problem}

According to Ref.\ \cite{cjeDiss}, the problem is a straightforward one.

First, we take the well-known equation for the Rabi frequency, 

\begin{equation}
\Omega = \frac{-eE_0}{\hbar}\langle e |\vec{r}|g\rangle=\frac{\vec{d}E_0}{\hbar}
\end{equation}

Then, we write 

\begin{equation}
\Omega_\mathit{eff}=\sqrt{\Omega^2+\delta^2}
\end{equation}

Then, we see that 
\begin{equation}
\Omega_\mathit{eff}=\sqrt{\left(\frac{\Gamma^2S_0}{2}\right)^2 + \delta^2}
\end{equation}

Then, according to \cite{RamanBeamSplit} and \cite{footAtomicPhysics}, we can say that 
\begin{equation} \label{KorsunskysJewel}
\Omega_\mathit{Raman}=\frac{2\Omega_1\Omega_2}{\Delta}
\end{equation}

\section{Background}

First, we write the Hamiltonian of the system. We neglect all except the following states: %why does the ground state J not play into it at all?
%I just realized I have no idea where to find the information on the hyperfine splitting of the Sr+ 
%also, IDK what to do about the the Nuclear spin numbers of the upper state. 
%I have this http://link.springer.com/article/10.1007%2FBF00568145
First, we will write the complete Hamiltonian of the system. The $^2$S states can be rewritten as a sum of eigenstates of the $\vec{J}$ and $\vec{F}$ operators. 
There are 6 relevant states: 
\begin{center}
\begin{tabular}{|l|l|||r|}
\hline
$F=4$ $^2$S$_{1/2} (5s)$ & Ground state & label \\ \hline
$F=5$ $^2$S$_{1/2} (5s)$ & Ground state & label \\ \hline
$F=3$ $^2$P$_{3/2}$ ($5p$) & Intermediate state & label \\ \hline
$F=4$ $^2$P$_{3/2}$ ($5p$) & Intermediate state & label \\ \hline
$F=5$ $^2$P$_{3/2}$ ($5p$) & Intermediate state & label \\ \hline
$F=6$ $^2$P$_{3/2}$ ($5p$) & Intermediate state & label \\ \hline
\end{tabular}
\end{center}

The Hamiltonian will be the sum 


Then, the new Hamiltonian, $H$, is given by 

\begin{equation}
H=H_0+V
\end{equation}
rotating wave approximation (RWA). Let $H_0$ be the Hamiltonian of our unperturbed system (the atom) and let $V$ be the perturbation (which we will use to model the electric field). 


Now, we can move to the interaction picture in the standard way. Let $|\alpha\rangle$ be the eigenstate of $H_0$, the unperturbed Hamiltonian: 

\begin{equation}
|\alpha\rangle_I=e^{-iE_\alpha t/h}|\alpha\rangle
\end{equation}

Then, 
%I don't know whether to use I or J. I mean, are there now tons of terms? 
\begin{align}
-i\hbar \frac{d}{dt}|n\rangle = 
\end{align}

\section{Finding the Dipole Moment Matrix Operator}
We would like to carefully determine the value of this. According to Ref.\ \cite{safronova2photon} the magnitude of the dipole moment operator is -4.35075 a. u. \footnote{atomic units, $a_0 e$} as calculated using the all-order, relativistic SD method. It is useful to compare this to the value obtained from at least one other source \footnote{I want to make sure the units and conventions are what I think they are.}. According to the NIST atomic spectra database, $A_{ki}=1.41e8$ s$^{-1}$ \cite{NISTasd}. This seems to be the Einstein A coefficient associated only with the decays from this state. If this is the case, then we use this equation from Ref.\ \cite{demilleBudkerKimball}:  
\begin{equation}
|\langle f ||d|| i \rangle|^2 = (4 \pi \epsilon_0) \frac{3 \hbar c^3}{4 \omega_0^3} (2 J'+1) A_{ki}\label{budkerAeqn} 
\end{equation}

This comes from slightly modifying Equation 3.117. It was necessary to convert it from Gaussian units by taking $d\rightarrow d \sqrt{4 \pi \epsilon_0}$. Furthermore, what Ref.\ \cite{demilleBudkerKimball} calls $\gamma$ must be renamed $A_{ki}$. Here $J'$ refers to the total angular momentum of the electron in the upper state, which in our case is $3/2$. Plugging in our values into Ref.\ \ref{budkerAeqn}, we get that the magnitude of $|\langle f ||d|| i \rangle|$ is \href{http://www.wolframalpha.com/input/?i=sqrt%283*hbar*c%5E3%2F%284*%282*pi*c%2F407.771+nm%29%5E3%29*4*pi*epsilon_0*4*1.41e8*1%2Fs%29}{4.344 electron Bohr Dipole Moments}.
Thus, the agreement between the theoretical calculations of Ref.\ \cite{safronova2photon} and the experimentally-derived values of Ref.\ \cite{NISTasd} is very good.  

Based on the looks of this, it seems like we already have the reduced matrix element. 


This is doubly true because DeMille et. al. already used the Wigner Eckhart theorem to get this. \footnote{But did they use the Wigner Eckhart theorem to combine the different states or did they use it to split them out?} 

From \cite{demilleBudkerKimball}, we see that the value for $\langle f||d||i\rangle$ that we have obtained is the reduced dipole matrix element. In order to apply it to any one of our given states, we would need to apply the wigner-Eckhart theorem. 

So, for example, to find the coupling between the $^2 P_{3/2}$ state with $m_j=+3/2$, we would just take 

\begin{equation}
\langle ^2 P _{3/2}|T_q^{(k)}|^2 S_{1/2}\rangle = 
\end{equation}

%crap--what about spin? 

\section{Finding the Density Matrices to describe the system}


Now, we explicitly neglect any coupling there may be between the external laser field and the nucleur spin. We can write the laser thing like this: 

%what is conserved? 

We will write the states in terms of eigenvectors of the following: $J$, $L$, $s$, $F$, $m_j$.

Oh, it seems like selection rules will tell us the answer. From $F=5$, we can go to $F=6$, from there, we can go to what? 

%Oh, it seems like even though the external light field doesn't couple to the nucleus directly, it does couple other things, which can be rewritten in a basis etc. and get coupled themselves. 

%Oh, also, it seems like maybe the upper state, we are not looking at all possible values of $L$.

%Do I need to worry about Doppler broadening and all that? 
Importantly, note that once we complete a perturbation, we just end up with a new Hamiltonian that takes a diagonalized form:

\begin{equation}
\begin{pmatrix}
E_1 & 0 & \cdots & 0 \\
0   & E_2 & \cdots & 0\\
\vdots&\vdots&\ddots & \vdots\\
0 & 0 & \cdots & E_n
\end{pmatrix}
\end{equation}

We will then rewrite the Hamiltonian in terms of the total angular momentum. 

How do we deal with multiple things? I mean, why does the multiplicity of states make it more likely to end up in those states? 

OK, suppose that you made a mistake and put the same state in the system twice. Then, it starts to become very important to count the right number of states. But now, suppose there was something you didn't know about, like the hyperfine structure. This splits all the states, I guess. 

Sure, now perturbations split states all the time. But can a perturbation split the states just by existing? By making two states where there was one, I mean? 

Yes, of course. This is called a direct product. Everything is direct products. Really, the way to look at it is that the atom is a simple system. It is a direct product of a bunch of different Hamiltonians. Then these Hamiltonians are coupled in particular ways. You can really just throw away all the intuition that comes from the actual meaning of the quantum numbers. 


We see that we already have the coupling constants for most of them. %Wait--how does the hyperfine thing come into it? Does it at all? 

\begin{align}
\langle ^2 P_{3/2}||\mathbf{d}||^2 S_{1/2} \rangle &= 4.35 \\
\langle ^2 P_{2}||\mathbf{d}||^2 S_{1/2} \rangle &= 4.35
\end{align}

We can then treat it perturbatively. 

We will then justify using this equation \cite{footAtomicPhysics} \cite{RamanBeamSplit}: 

\begin{equation}
\Omega_{\rm eff}=\frac{\Omega_{2i} \Omega_{i1}}{2 \Delta}
\end{equation}
where $\Omega_{2i}$ and $\Omega_{i1}$ are the Rabi frequencies corresponding to the transition from our $^2$S$_{1/2}$ ground state to the  $^2$P$_{3/2}$ state. 
The key issue with this segment of the paper is that I want to make sure that we're doing the right thing with the dipole moment operators and the modelling of the atomic transition.  %what about the width of the transition? Ah, that should all come out of the soup. %basically, the dominant effect is . . . something? It's the dipole stuff and the tuning, etc., not an energy thing? well, no, it's just that changing the absolute energy doesn't matter much.  
% It has just occurred to me (today, 2013-130) that we are doing the right thing. Well, I mean, in reality, the coupling of our two states to our lasers will be the same because the orbital angular momentum state is what matters for the purposes of calculating the coupling of an external field to our atom. 

% Well, it has just occurred to me that I don't know about the angular momentum conservation. It is conserved, but strangely. 

% So, wait--are there 0-0 transitions? Yes--one photon adds the angular momentum, the other one takes it away. 

 I would very much like to see the forbidding of the non-angular-momentum-conserving transitions come out of the soup, as they say. I mean, I would like to see a Hamiltonian that does not couple them at all. In addition, I would like to see the relationship between the magnetic field we have placed and see which direction the angular momentum from our lasers must go. 

%I have very much been thinking about symmetries and ways to rewrite things. Like, if the additive inverse of 0100111 were encoded as 1011000. Then you could store same amount of information, but with different encoding. One way to encode would be to merely find 7 symmetry operators, I think. Such as what? I'm not sure.

So write that Hamiltonian.  

%Ah, so how do we make sure that our lasers both add angular momentum in the same way? Does it matter how they're oriented relative to the magnetic field? 

The Rabi frequencies are given by 

\begin{equation}
\Omega=\frac{\mu E}{h}
\end{equation} \cite{hilbornNoGetConfused}
where $\mu$ is the dipole moment matrix element describing the coupling between the electric field and the atom. 

Hilborn gives an expression for $\mu^2$ in terms of the oscillator strength, which I've found in two places \cite{safronovaTheory} \cite{NISTasd} to be about .7. However, I think I need to apply the Wigner-Eckhart theorem as explained in \cite{demilleBudkerKimball}. This also matches Gallagher's answer. 

We carefully make sure that we're using the right dipole moment matrix elements. This is something I still need to do. 

We can compare our result to Chris' thesis. 


  \section{Calculation of ideal intensities}

We use the derived formula above to calculate the necessary beam geometries. We can compare this to Chris' thesis. 


