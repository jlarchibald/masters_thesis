% % These packages typeset the thesis with Times Roman font
% \usepackage[T1]{fontenc}
% \usepackage{textcomp}
% \usepackage{mathptmx}
\usepackage{dsfont}
\usepackage{blkarray}
%I love these packages.
\usepackage{amsmath}
\usepackage{amssymb}


% Allow the inclusion of graphics
\usepackage{graphicx}
%\usepackage{subfigure}
\usepackage{pdfpages}

% The fancyhdr package allows you to easily customize the page header.
% The settings below produce a nice, well separated header.
\usepackage{fancyhdr}
  \fancyhead{}
  \fancyhead[LO]{\slshape \rightmark}
  \fancyhead[RO,LE]{\textbf{\thepage}}
  \fancyhead[RE]{\slshape \leftmark}
  \fancyfoot{}
  \pagestyle{fancy}
  \renewcommand{\chaptermark}[1]{\markboth{\chaptername \ \thechapter \ \ #1}{}}
  \renewcommand{\sectionmark}[1]{\markright{\thesection \ \ #1}}

% The caption package allows you to change the formatting of figure captions.
% The commands here change to the suggested caption format: single spaced and a bold tag
\usepackage[margin=0.3in,labelfont=bf,labelsep=none]{caption}
 \DeclareCaptionFormat{suggested}{\singlespace#1#2 #3\par\doublespace}
 \captionsetup{format=suggested}

%Added by James Archibald 2015-10-03.
%This package is the current preferred way to make subfigures. It seemed to give me trouble
%when I tried to put this statement before the \usepackage{caption} statement.
\usepackage{subcaption}


% The cite package cleans up the way citations are handled.  For example, it
% changes the citation [1,2,3,6,7,8,9,10,11] into [1-3,6-11].  If your advisor
% wants superscript citations, use the overcite package instead of the cite package.
\usepackage{cite}

% The makeidx package makes your index for you.  To make an index entry,
% go to the place in the book that should be referenced and type
%  \index{key}
% An index entry labeled "key" (or whatever you type) will then
% be included and point to the correct page.
\usepackage{makeidx}
\makeindex


\usepackage{epstopdf}
\usepackage{cancel}

% If you have a lot of equations, you might be interested in the amstex package.
% It defines a number of environments and macros that are helpful for mathematics.
% We don't do much math in this example, so we haven't used amstex here.

% The hyperref package provides automatic linking and bookmarking for the table
% of contents, index, equation references, and figure references.  It must be
% included for the BYU Physics class to make a properly functioning electronic
% thesis.  It should be the last package loaded if possible.
%
% To include a link in your pdf use \href{URL}{Text to be displayed}.  If your
% display text is the URL, you probably should use the \url{} command discussed
% above.
%
% To add a bookmark in the pdf you can use \pdfbookmark.  You can look up its usage
% in the hyperref package documentation
%\usepackage[bookmarksnumbered,pdfpagelabels=true,plainpages=false,colorlinks=true,
            %linkcolor=black,citecolor=black,urlcolor=blue]{hyperref}
%\urlstyle{rm}
% ------------------------- Fill in these fields for the preliminary pages ----------------------------
%
% For Senior and honors this is the year and month that you submit the thesis
% For Masters and PhD, this is your graduation date
  \Year{2015}
  \Month{December}
  \Author{James Lawrence Archibald II}

% If you have a long title, split it between two lines using the \\ command.
% A multiple line title should be an "inverted pyramid" with the top line(s) longer than the bottom.
  \Title{Construction of a 408 nm Laser System \\ for Use in Ion Interferometry}

% Your research advisor
\AdvisorTitle{Chair}
  \Advisor{Dallin S. Durfee}
%% The members of your committee (masters only need A and B, PhD need all 4)
  \MemberA{Scott Bergeson}
  \MemberB{Jean-Francois VanHuele}
% The text of your abstract
  \Abstract{
This work reports on the construction of a 408 nm laser system designed to drive stimulated Raman transitions between the $F=4$ and $F=5$ $^2$S$_{1/2}$ states of $^{87}$Sr$^+$ using the $^2$P$_{3/2}$ state as the intermediate state. This laser system will be used as part of a $^{87}$Sr$^+$ ion interferometer. This work also includes a discussion of relevant theory describing the interaction of the ions and laser, along with a calculation of the transition rates as a function of laser power and detuning.
}

 \Keywords{Sr, ion interferometer, stimulated Raman transition}

% Acknowledge those who helped and supported you
  \Acknowledgments{
I have benefitted enormously from the kindness and generosity of many people who went above and beyond what they were expected to do to support me in the writing of this thesis.

I am grateful to the entire BYU physics department and faculty. Even though I was not always a model student, I loved my time at BYU and I benefitted enormously from the people and facilities I encountered there. The staff and faculty at BYU always made me feel like the system was on my side, even when I did not deserve it. Many professors went the extra mile to help me out or give me advice. I never felt like an imposition or a burden, even when I was, and I frequently felt surprised at the level of personal attention and care that I received at the hands of my instructors. In particular, I would like to thank the three longstanding members of my committee--Drs. Bergeson, Van Huele and Durfee for their patience and for their embodiment of this ideal in my experiences with them in the classroom and as I have come to them for help with the various problems I encountered in my research and learning.

I am particularly indebted to my advisor, Dallin. The level of patience and support he has shown me have been orders of magnitude more than I ever could have hoped for or expected. He has been there for me during dark periods of my life. He is also a model teacher, scholar and friend. 

This thesis would not have happened without the help and support of my lovely wife, Camille, whose deeply-felt desire to see me succeed in finishing this thesis has moved me in a way that few things have. I would also like to thank my parents for their unyielding support and my old boss Kerry Glover.
  }

\fussy
\DeclareMathOperator{\erf} {erf}
