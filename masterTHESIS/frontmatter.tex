% % These packages typeset the thesis with Times Roman font
% \usepackage[T1]{fontenc}
% \usepackage{textcomp}
% \usepackage{mathptmx}

%I love these packages.
\usepackage{amsmath}
\usepackage{amssymb}


% Allow the inclusion of graphics
\usepackage{graphicx}
\usepackage{subfig}

% The fancyhdr package allows you to easily customize the page header.
% The settings below produce a nice, well separated header.
\usepackage{fancyhdr}
  \fancyhead{}
  \fancyhead[LO]{\slshape \rightmark}
  \fancyhead[RO,LE]{\textbf{\thepage}}
  \fancyhead[RE]{\slshape \leftmark}
  \fancyfoot{}
  \pagestyle{fancy}
  \renewcommand{\chaptermark}[1]{\markboth{\chaptername \ \thechapter \ \ #1}{}}
  \renewcommand{\sectionmark}[1]{\markright{\thesection \ \ #1}}

% The caption package allows you to change the formatting of figure captions.
% The commands here change to the suggested caption format: single spaced and a bold tag
\usepackage[margin=0.3in,labelfont=bf,labelsep=none]{caption}
 \DeclareCaptionFormat{suggested}{\singlespace#1#2 #3\par\doublespace}
 \captionsetup{format=suggested}


% The cite package cleans up the way citations are handled.  For example, it
% changes the citation [1,2,3,6,7,8,9,10,11] into [1-3,6-11].  If your advisor
% wants superscript citations, use the overcite package instead of the cite package.
\usepackage{cite}

% The makeidx package makes your index for you.  To make an index entry,
% go to the place in the book that should be referenced and type
%  \index{key}
% An index entry labeled "key" (or whatever you type) will then
% be included and point to the correct page.
\usepackage{makeidx}
\makeindex

% If you have a lot of equations, you might be interested in the amstex package.
% It defines a number of environments and macros that are helpful for mathematics.
% We don't do much math in this example, so we haven't used amstex here.

% The hyperref package provides automatic linking and bookmarking for the table
% of contents, index, equation references, and figure references.  It must be
% included for the BYU Physics class to make a properly functioning electronic
% thesis.  It should be the last package loaded if possible.
%
% To include a link in your pdf use \href{URL}{Text to be displayed}.  If your
% display text is the URL, you probably should use the \url{} command discussed
% above.
%
% To add a bookmark in the pdf you can use \pdfbookmark.  You can look up its usage
% in the hyperref package documentation
\usepackage[bookmarksnumbered,pdfpagelabels=true,plainpages=false,colorlinks=true,
            linkcolor=black,citecolor=black,urlcolor=blue]{hyperref}
\urlstyle{rm}
% ------------------------- Fill in these fields for the preliminary pages ----------------------------
%
% For Senior and honors this is the year and month that you submit the thesis
% For Masters and PhD, this is your graduation date
  \Year{2015}
  \Month{July}
  \Author{James Lawrence Archibald II}

% If you have a long title, split it between two lines using the \\ command.
% A multiple line title should be an "inverted pyramid" with the top line(s) longer than the bottom.
  \Title{Construction of a 408 nm Laser System \\ for use in Ion Interferometry}

% Your research advisor
  \Advisor{Dallin S. Durfee}

% The text of your abstract
  \Abstract{
We discuss the construction of a 408 nm laser system for use in a $^{87}$Sr+ ion interferometer. This work reports progress towards driving stimulated Raman transitions between the $F=4$ and $F=5$ $^2$S$_{1/2}$ states of $^{87}$Sr$+$ using the $^2$P$_{3/2}$ state as the intermediate state.

The scope of this work is limited to the construction of a laser system that is intended to drive the stimulated Raman transitions in our ion interferometer experiment (also known as the 408 nm 5GHz detuned laser system).
This work also includes a discussion of the relevant theory along with some calculations about the transition rates and power levels involved in getting the experiment to work. 

We will also provide a brief summary of the ion interferometer and a discussion of its place among related experiments. 
}

 \Keywords{Sr, ion interferometer, photon rest mass, laser, AOM}

% Acknowledge those who helped and supported you
  \Acknowledgments{Camille has been a wonderful support. 
  }

\fussy
\DeclareMathOperator{\erf} {erf}
